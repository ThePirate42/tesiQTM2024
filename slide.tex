\documentclass{beamer}

\mode<presentation>
{
	\usetheme[
		titlepagelogo=logo_verticale_BLACK.png,% Logo for the first page
		language=italian,
		coding=utf8,
		bullet=square,
		pageofpages=di,
		color=blue,
		secondsupervisor=true,
	]{TorinoTh}
	
}

\usepackage[italian]{babel}
\usepackage{mathtools}

\title
{Macchine di Turing Quantistiche}

\author
{Pietro Zignaigo}

\institute
{Università di Genova}

\rel
{Elena Zucca}
\secondsupervisor
{Francesco Dagnino}

\date
{16-12-2024}

\subject
{Macchine di Turing Quantistiche}
% This is only inserted into the PDF information catalog. Can be left
% out.


% If you wish to uncover everything in a step-wise fashion, uncomment
% the following command: 

%\beamerdefaultoverlayspecification{<+->}


\begin{document}

\begin{frame}
	\titlepage
\end{frame}

\begin{frame}
	\tableofcontents
\end{frame}

\section{Introduzione}

\subsection{Computazione quantistica}

\begin{frame}{\subsecname}
	\begin{itemize}
		\item \textbf{Quantum advantage}: A parità di problema, la complessità temporale degli algoritmi quantistici può essere minore di quella degli algoritmi classici.
		\item Lo stato di un computer quantistico è una sovrapposizione di stati discreti.
	\end{itemize}
\end{frame}

\begin{frame}{\subsecname}{}
	\begin{itemize}
		\item Per modellare uno stato quantistico si utilizzano gli \textit{spazi di Hilbert}:
	\end{itemize}
	\[ \ell^{2} \left ( \mathcal{B} \right ) = \left \{ \phi : \mathcal{B} \rightarrow \mathbb{C} \mid \sum_{\mathcal{C} \in \mathcal{B}} \left | \phi \left ( \mathcal{C} \right ) \right |^{2} < \infty \right \}\]
	\begin{itemize}
		\item Per ragioni fisiche, possono essere applicate a gli elementi dello spazio solo \textit{operatori unitari}:
		\begin{itemize}
			\item invertibili
			\item conservano la norma
		\end{itemize}
	\end{itemize}
\end{frame}

\subsection{Macchina di Turing}

\section{Macchina di Turing Quantistica}

\subsection{Q-configurazioni}

\subsection{Funzione δ}

\subsection{Macchina di Turing Quantistica}

\subsection{Condizioni di unitarietà}

\section{Quantum computable functions}

\subsection{Dominio e codominio}

\subsection{Categorie di terminazione}

\section{Misurazioni}

\end{document}