\documentclass{beamer}

\mode<presentation>
{
	\usetheme[
		titlepagelogo=logo_verticale_BLACK.png,% Logo for the first page
		language=italian,
		coding=utf8,
		bullet=square,
		pageofpages=di,
		color=blue,
		secondsupervisor=true,
	]{TorinoTh}
	
}

\usepackage[italian]{babel}

\title
{Macchine di Turing Quantistiche}

\author
{Pietro Zignaigo}

\institute
{Università di Genova}

\rel
{Elena Zucca}
\secondsupervisor
{Francesco Dagnino}

\date
{16-12-2024}

\subject
{Macchine di Turing Quantistiche}
% This is only inserted into the PDF information catalog. Can be left
% out.


% If you wish to uncover everything in a step-wise fashion, uncomment
% the following command: 

%\beamerdefaultoverlayspecification{<+->}


\begin{document}

\begin{frame}
	\titlepage
\end{frame}

\begin{frame}
	\tableofcontents
\end{frame}

\section{Introduzione}

\subsection{Computazione quantistica}

\begin{frame}{\subsecname}
	\begin{itemize}
	\item Use \texttt{itemize} a lot.
	\item Use very short sentences or short phrases.
	\end{itemize}
\end{frame}

\subsection{Macchina di Turing}

\section{Macchina di Turing Quantistica}

\subsection{Q-configurazioni}

\subsection{Funzione δ}

\subsection{Macchina di Turing Quantistica}

\subsection{Condizioni di unitarietà}

\section{Quantum computable functions}

\subsection{Dominio e codominio}

\subsection{Categorie di terminazione}

\section{Misurazioni}

\end{document}